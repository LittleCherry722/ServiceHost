\documentclass[11pt]{tudexercise}

\usepackage[utf8]{inputenc}
\usepackage{german}
\usepackage{hyperref}
\usepackage{color}

\title{Aufgaben zum Backend Tutorial}

%commands:
\newcommand{\myimport}{\textbf{import} }
\newcommand{\link}[1]{\\ \textcolor{blue}{\textit{\url{#1}}}}

\begin{document}
  \maketitle

\section{Akka lernen}
  Im ersten Schritt verweisen wir auf ein externes Beispiel.
  Folge diesem Link:
  \link{http://doc.akka.io/docs/akka/2.0/intro/getting-started-first-scala.html}\\
  und bearbeite das Tutorial. Im Git liegt unter exercises/AkkaTutorial ein leeres Scala-Projekt
  mit einem sbt script, deshalb kann der Teil des Tutorials, der sich mit dem Einrichten des Projekts
  beschäftigt, übersprungen werden.
  (Führt das Skript aus und importiert es von der Stelle in Eclipse, an der es liegt.
  Eure Lösungen sollt ihr nicht hochladen)

  \paragraph*{Hinweis:}
    Im Tutorial werden die imports:\\
    \myimport akka.util.Duration\\
    \myimport akka.util.duration.\_

    genutzt. Dies funktioniert mit der aktuellen akka Version nicht mehr, nutzt stattdessen den import:\\
    \myimport scala.concurrent.duration.\_

\section{Backend Tutorial}
  Das Backend Tutorial soll dazu dienen, die Konzepte des Backends an Hand einer vereinfachten
  Version besser zu verstehen.

  \subsection{Einrichten}
    Im Git-Ordner exercises/BackendTutorial liegt eine rudimentäre Implementierung des Backends,
    die den Einstieg in den Backendkernel vereinfachen soll.

    Zur Bearbeitung der Aufgaben empfehlen wir den HTTPRequester für Firefox
    \link{https://addons.mozilla.org/de/firefox/addon/httprequester/}\\
    oder Advanced REST Client für Chrome
    \link{https://chrome.google.com/webstore/detail/advanced-rest-client/hgmloofddffdnphfgcellkdfbfbjeloo}

    Die Struktur und eine Erklärung der einzelnen Komponenten ist unter diesem Link verfügbar:
    \link{https://sbpm-groupware.atlassian.net/wiki/display/SBPM/SBPM+-+Tutorial}\\
    Kopiere den Ordner BackendTutorial nach exercises/tutorial/<lastname>/ und führe anschließend das sbt Skript aus.
    Importiere das BackendTutorial als Projekt in die Entwicklungsumgebung und starte die Ausführung
    über die Boot.scala.

    Versuche über GET http://localhost:8080/subject/1 den Status des ersten Subjekts auszulesen (es gibt 2).

  \subsection{Fehler beheben}
    Ein ActState ist eine Verzweigung, die auf mehrere mögliche Folgestates zeigt. Momentan wird jedoch bei
    einer ExecuteAction-Nachricht nur die erste Transition ausgeführt. Erweitere die ExecuteAction-Nachricht
    und den ActStateActor so, dass es möglich ist, sich zwischen verschiedenen Verzweigungen zu entscheiden.

  \subsection{REST Schnittstelle erweitern}
    Bisher wurde immer das erste TestPair aus den Testdaten genommen, um die beiden Subjekte zu instantiieren.
    Nun soll die Möglichkeit geschaffen werden auch das zweite Testpair zu instantiieren,
    hierfür existiert bereits eine Funktion im ProcessInstanceActor,
    um die beiden Subjekte auszutauschen.

    Als Aufruf soll der Befehl:
    \textbf{PUT} http://localhost:8080/subject\\
    mit dem Argument: \textbf{\{ “instance”: n \}} dienen,\\
    wobei n ein Int ist und das Testbeispiel auswählen soll.\\
    Hierfür muss die REST-Schnittstelle angepasst und eine neue Nachricht eingeführt werden.

  \subsection{Ausführung erweitern}
    In dieser Aufgabe soll eine einfache Kommunikation zwischen zwei Subjekten realisiert werden.
    Die Kommunikation besteht aus SendStates , die eine Nachricht an das andere Subjekt schicken
    und auf eine Bestätigung\footnote{implementiert als \textbf{case object} Ack}(Ack) warten um den State zu wechseln,
    während ReceiveStates\footnote{Der ReceiveState ist bereits implementiert}
    auf Nachrichten warten, den Erhalt bestätigen und nach dem dann in der Lage sind
    in den nächsten State zu wechseln. Hierzu soll das zweite Testpair verwendet werden,
    das im Gegensatz zum 1. auch aus Receive- und Sender-States besteht.

\end{document}
